\par{This OpenCL implementation of the LU decomposition was done with three 
    separate \emph{kernels}. Each of these \emph{kernels} performed operations on different 
    areas of a NxN matrix and the \emph{kernels} were enqueued multiple times.}

\par{A key parameter in this algorithm is the block size, which is defined 
    both in the host and \emph{kernel} code. It can take values of 2, 4, 8, 16, 32 
    and 64. A block size of 64 is not supported on the GPU, due to \emph{work group} 
    size restrictions. The value of the block size is not changed during the 
    computation. The block size explicitly determines a number of variables, 
    including the number of \emph{work groups}, the number of \emph{work items} per 
    \emph{work group}, the size of the data block and the number of times the 
    \emph{kernels} are enqueued. It is also important to note that the overall number of 
    floating point operations required to compute the LU decomposition varies 
    according to the block size.}


\par{The three \emph{kernels} are enqueued at each iteration of a for loop in the 
    host code, with various parameters such as the number of \emph{work groups}, 
    number of \emph{work items} and data size changing at each iteration of the 
    loop.}

