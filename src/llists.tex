\par{Verlet neighbour lists, or as they are commonly known linked lists, are a common method in 
molecular dynamics codes to reduce drastically the number of force evaluations
during the time propagation\cite{Verlet1997}. DL\_POLY implements a modern version
of these lists\cite{Pinches1991,Smith1994,Hockney1981}. The main idea is the partitioning of the entire simulation
space into cells such that in order to determine all the atoms with which one atom interacts one needs to check 
only the neighbouring cells. The version implemented in DL\_POLY is efficient with respect to the 
MPI processes, see \fref{fig:base} pink curve, over 70\% for both Gramidicin and Iron on Host. This picture 
is not replicated in the case of the Xeon Phi.}

\par{OpenMP parallelisation of the most time consuming loop involved an extensive refactoring of the original 
implementation. Instead of looping over all the cells and then over all the atoms in a particular cell 
we loop over all the atoms. In this way we increas the amount of work that can be done in parallel. The results, 
presented in \fref{fig:mpi1}, show a good scalability for small thread counts in the case of Iron on both Host 
and Xeon Phi. Unfortunately, not the same behaviour is replicated in the case of Gramidicin. The poor scalability 
can be attributed to the multiple branching within the OpenMP region and the role of the sequential code from this segment.}
