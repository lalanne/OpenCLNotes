\par{This code segment is executed for every time step of the simulation(only in the case of the Iron system) and it is used to 
compute local densities in metals, this is performed using the verlet neighbour list and Finnis-Sinclair type potential 
\citep{finnis1984sen}. These local densities are computed sequentially using two loops over the atoms of every MPI process.}

\par{The parallelisation of this algorithm was performed using OpenMP over these two loops, the results of this parallelisation
are shown by the green curve on the bottom panels of \fref{fig:mpi1}. This parallelisation does not achieve good efficiency, with 
44\% with 20 threads on the Host and 17\% with 60 threads on the Xeon Phi. The main reasons for this poor performance is an 
MPI process synchronization segment at the end of the computation of metal densities and some sequential code still present in
the computation.}

\par{We noticed during the parallelisation of this code that some calculations performed are also computed in 
two body forces \emph{eg.} calculation of the differences between atom positions. To optimise this, we stored these values to not 
perform the same computation twice. We did not obtain good results with this optimisation, due to the increase of memory foot 
print that the implementation of this optimisation introduced, also the code added in two body forces to allow this optimisation 
degraded the performance in the case of the Gramidicin system.}

