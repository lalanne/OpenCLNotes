\par{As the name suggests, the diagonal \emph{kernel} performs operations on the 
    diagonal elements of the matrix. Table \ref{tab:lu1} summarises how the block 
    size affects the implementation of the diagonal \emph{kernel}.}

\begin{table}[!h]
    \centering
    \begin{tabular}{| c | c | c | c |}
    \hline
    \emph{Block Size} & \emph{Data Block} & \emph{\#Work Groups} & 
                    \emph{\#Work-Items / Work-Group} \\ \hline
    2 & 2x2 & 1 & 2 \\ \hline
    4 & 4x4 & 1 & 4 \\ \hline
    8 & 8x8 & 1 & 8 \\ \hline
    16 & 16x16 & 1 & 16 \\ \hline
    32 & 32x32 & 1 & 32 \\ \hline
    64 & 64x64 & 1 & 64\\ \hline
    \end{tabular}
    \caption{Diagonal \emph{kernel} summary.}
    \label{tab:lu1}
\end{table}

\par{The first point to note is that the diagonal \emph{kernel} is clearly not 
    optimised for parallel computing. In OpenCL, the \emph{work groups} are scheduled 
    in parallel on each thread of the Xeon Phi and Xeon CPU. In this case, 
    the number of \emph{work groups} is always one. For example, on the Xeon Phi, 
    the entire computation of the diagonal \emph{kernel} would take place on just one 
    of the 240 hardware threads.}

\par{Although there is no parallelism between threads in this case, 
    implicit SIMD vectorisation of \emph{work items} is supported by the OpenCL 
    compiler. However, this requires both that \emph{kernels} be as simple as possible 
    and that they execute the same instructions. This is not the case with this 
    \emph{kernel}. The algorithm is such that each \emph{work item} carries out a different
    number of floating point operations, depending on its ID within the 
    \emph{work group}. This renders instruction-level parallelism impossible in this 
    case. Figure \ref{FlopCount} shows that, for a \emph{work group} of 16 \emph{work items}, 
    the number of floating point operations varies according to the ID of the 
    \emph{work item}. In fact, the first \emph{work item} of every \emph{work group} does no 
    computation whatsoever.}

\begin{figure}[!h]
    \centering
    \includegraphics[width=0.6\textwidth]{figures/FlopCount.png}
    \caption{Diagonal \emph{kernel}, flop count per \emph{work item}.}
    \label{FlopCount}
\end{figure}

\par{Figure \ref{DiagonalKernel} illustrates a number of important points. 
    It is clear that, out of the three devices used, the Xeon Phi shows 
    the worst performance. Achieving high-performance with the Xeon Phi 
    requires that \emph{work items} be vectorised and that large numbers of 
    \emph{work groups} be scheduled in parallel. Neither such condition is satisfied 
    in this case. It is interesting to note that the Xeon CPU performs best 
    with an algorithm of low parallelism. Undoubtedly, the CPU’s support for 
    out-of-order execution gives it an advantage over the Xeon Phi, which 
    supports in-order execution only.}

\begin{figure}[!h]
    \centering
    \includegraphics[width=0.6\textwidth]{figures/DiagonalKernel.png}
    \caption{4096x4096 diagonal \emph{kernel} times.}
    \label{DiagonalKernel}
\end{figure}

\par{The GPU also achieves good performance compared to the Xeon Phi. 
    Unlike on the Xeon CPU and Xeon Phi, the \emph{work items} are scheduled in 
    parallel in the warps of the GPU. It does not suffer the same lack of 
    performance as a result of having only 1 \emph{work group}.}



