\par{Listing \ref{more_work_kernel} its our attempt to give every 
    \emph{work item} more work, that way decreasing the effects of scheduling 
    and spawning overhead of \emph{work items} and \emph{work groups},
    this way the behaviour when cycles per \emph{work item} are increased can 
    be studied. This was achieved changing the dimension of the \emph{NDRange} 
    from 2 to 1 and making that every instance of a \emph{kernel} calculate the 
    results of one row of the resulting matrix, as it can be 
    seen in line \emph{10} of listing 4.}

\begin{figure}[!h]
    \centering
    \includegraphics[width=0.49\textwidth]{figures/opt1_phi.png}
    \includegraphics[width=0.49\textwidth]{figures/opt1_cpu.png}
    \includegraphics[width=0.49\textwidth]{figures/opt1_gpu.png}
    \caption{Results of matrix multiplication \emph{kernel} with more work per 
            \emph{work item} in different architectures.}
    \label{MoreWork}
\end{figure}

\par{Figure \ref{MoreWork} shows that in the case of the Xeon Phi co-processor 
    and the Xeon CPU, the effect of vectorization
    is the same that in the previous kernel(16 for the Xeon Phi and 4 for the 
    Xeon CPU), the other interesting effect in this
    case is the performance degradation after \emph{work group} dimension 32 
    on the Xeon Phi and after 128 on the Xeon
    CPU, clearly the Xeon Phi is more sensible to the decreasing number of 
    \emph{work groups} than the Xeon CPU, one of the 
    reasons for this is the number of \emph{computation units} available in 
    these 2 architectures, on the Xeon Phi is 236 and
    on the Xeon CPU is 40, as the number of \emph{work groups} decreases, 
    as it is shown on table \ref{tab:work_groups}, the amount
    of available parallelism that take advantage of the \emph{comput units} 
    available in these 2 architectures decreases as well.}

\par{{\color{red} ANALYSIS OF THE GPU....}}

\begin{table}[!h]
    \centering
    \begin{tabular}{| l | l | l | l |}
    \hline
    \emph{Work Group} Dimension & \#\emph{Work Groups} \\ \hline
    16 & 256 \\ \hline
    32 & 128 \\ \hline
    64 & 64 \\ \hline
    128 & 32 \\ \hline
    256 & 16 \\ \hline
    512 & 8 \\ \hline
    1024 & 4 \\ \hline
    2048 & 2 \\ \hline
    4096 & 1 \\ 
    \hline
    \end{tabular}
    \caption{\emph{Work group} dimension versus numbers of \emph{work groups}.}
    \label{tab:work_groups}
\end{table}

\par{In this case the most performant of the 3 architectures analized is the 
    Xeon as its shown on figure\ref{MoreWorkRes} {\color{red}whyyyyy??????}.}

\begin{figure}[!h]
    \centering
    \includegraphics[width=0.49\textwidth]{figures/moreWorkRes.png}
    \caption{Comparison between the best cases of the more work \emph{kernel} in different devices.}
    \label{MoreWorkRes}
\end{figure}

\par{Figures \ref{gpu}, \ref{phi} and \ref{xeon} show that the 
    only device that obtained some performance improvements from this \emph{kernel}
    change was the Xeon.{\color{red}whyyyyy??????}}





