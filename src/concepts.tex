
\begin{itemize}
    \item\textbf{\textit{Kernel:}} Function that it is going to be executed on an OpenCL \emph{device}\cite{opencl12}. 

    \item\textbf{\textit{Work Item:}} Kernel instance. A \emph{work item} is executed by one or more \emph{processing elements} as 
        part of a \emph{work group} executing on a compute unit\cite{opencl12}. It also could be defined as the basic unit of work
        on a OpenCL device\cite{intro_opencl}.

    \item\textbf{\textit{Work Group:}} Collection of related \emph{work items} that executes in a single \emph{compute unit}. The 
        \emph{work items} in a \emph{work group} executes the same kernel\cite{opencl12}.

    \item\textbf{\textit{Compute Unit:}} An OpenCL \emph{device} has a group of \emph{compute units}. A \emph{work group} always
        executes on a \emph{compute unit}, also a \emph{compute unit} is composed by one or more 
        \emph{processing elements}\cite{opencl12}(\emph{e.g. Streaming Multiprocessor in K20x Nvidia GPU, 
        hardware thread on Xeon Phi}).

    \item\textbf{\textit{Processing Element:}} A virtual scalar processor. A \emph{work item} executes in one or more 
        \emph{processing elements}\cite{opencl12}(\emph{e.g. vector lane in a CPU, thread in a GPU }). 

    \item\textbf{\textit{Platform:}} A \emph{platform} is a group of \emph{devices} managed by the Opencl framework 
        that allow an application to share resources and execute \emph{kernels} on \emph{devices} in the 
        \emph{platform}\cite{opencl12}.

    \item\textbf{\textit{Device:}} Collection of \emph{compute units} \emph{e.g. GPUs, multi-core CPU, DSPs or Cell/B.E. processor}
        \cite{opencl12}.
\end{itemize}




