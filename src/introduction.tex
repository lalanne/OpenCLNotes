\par{Molecular dynamics techniques grew rapidly in the last twenty years. 
The grow was fuelled by development of new scalable mathematical algorithms, availability 
of powerful hardware and better availability of ready to use software packages. DL\_POLY is one of 
these packages, widely adopted by the computational physics and material science communities.}
\par{DL\_POLY started its life in 1994 at Daresbury Laboratory, now part of Science \& Technology Facilities 
Council in United Kingdom. The main developers for the currect version are W Smith and IT Todorov. DL\_POLY is 
a general classical molecular dynamics code and was used to simulate macro molecules (both biological and synthetic),
complex fluids, materials and ionic liquids. DL\_POLY also plays an important role as sandbox for both development
of new methods and algorithms for molecular dynamics and testing of emerging hardware 
technologies\cite{dlpoly} and \cite{Todorov2006}.
The core code is written in Fortran 90 and optimised for distributed systems using domain decomposition, also OpenMP and CUDA ports exist as 
contributions to DL\_POLY but not part of the official distribution.
DL\_POLY is free of use for UK academics pursuing non-commerical research and available for licensing for the rest.
Over 3000 licenses were offered over the years worldwide.
}
\par{The Intel Xeon Phi co-processor is a novel accelerator technology that provides few appealing features as:
many cores, 60 cores with 240 hardware threads for the mid model, low power consumption, the same set as 
instructions as an Intel CPU, supports popular and standardised programming models as MPI and OpenMP and a 
theoretical peak of 1~TFlops.}
\par{In this communication we present the progress made in porting and optimising DL\_POLY to Xeon Phi co-processor. 
The rest of the paper is organised as follows: a short introduction to the methodology used for port and optimisation,
OpenMP implementation results in \sref{sec:openmp}, synchronous offload ones in \sref{sec:offload}, MPI symmetric
 running mode in \sref{sec:mpis}.}
