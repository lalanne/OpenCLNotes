\par{Shake is the name of the algorithm for the treatment of dynamical systems with holonomic constraints  
\cite{Ryckaert1977}. DL\_POLY uses a modified version of the original algorithm as described in \cite{Smith1994a}. 
Shake is originally a sequential iterative algorithm, the version implemented in DL\_POLY is modified to allow distribution over 
MPI processes. Each iteration involves synchronization of the coordinates of atoms in all the MPI processes which contain
constraints. This makes the OpenMP parallelisation a challenging task.}

\par{We parallelised for each shake iteration the most intensive loops, unfortunatelly the last loop of the algorithm in which 
the atomic positions are corrected contains six atomic operations which cannot be removed due to the fact that the 
parallelisation is made over constraints and one atom can belong to more that one constraint. In the process of profiling(vtune) 
we discovered that the division operation is atipically expensive on the Xeon Phi hence we transformed some expressions wich 
contained it.}

\par{OpenMP scaling for Shake is excelent within the Host with an efficiency of 88\% for 20 threads and on the Xeon Phi we
achieved good scalabilty with a relative efficiency up to 74\% with 60 threads, as shown by the green curve  in the upper panels
of \fref{fig:mpi1}. Increasing further the thread count on Xeon Phi the efficiency degrades. This is due the fact that the 
algorithm does not benefit from the usage of the extra trheads on each core. Further investigations are required in order to 
increase the usage of the extra threads. In terms of absolute times the pure OpenMP implementation of Shake on the Xeon Phi is 
77\% slower than the pure MPI version when 10 MPI processes are used and are around 3 times slower compared with a full Host.}

\par{On the Xeon Phi the OpenMP implementation with one MPI process is 66\% faster than the best time for shake with pure MPI for 
the Gramidicin system.}

\par{Incrementing the number of MPI processes decreases the OpenMP efficiency of Shake \fref{fig:mpi2}, this is due to the MPI 
synchronization of data which needs to be done every iterative step of Shake.} 
